\documentclass{beamer}
%
% Choose how your presentation looks.
%
% For more themes, color themes and font themes, see:
% http://deic.uab.es/~iblanes/beamer_gallery/index_by_theme.html
%
\mode<presentation>
{
  \usetheme{Warsaw}      % or try Darmstadt, Madrid, Warsaw, ...
  \usecolortheme{beaver} % or try albatross, beaver, crane, ...
  \usefonttheme{default}  % or try serif, structurebold, ...
  \setbeamertemplate{navigation symbols}{}
  \setbeamertemplate{caption}[numbered]
} 

\usepackage[polish]{babel}
\usepackage[utf8x]{inputenc}
\usepackage[T1]{fontenc}
\usepackage{mathtools}
\usepackage{tikz}
\usetikzlibrary{positioning}
\usepackage{minted}

\title[Metody drzew czerwono-czarnych oraz drzew AVL]{Równoważenie binarnych drzew przeszukiwania \\ metodą drzew czerwono-czarnych\\ oraz metodą drzew AVL}
\author[Michał K. Feiler]{Michał Krzysztof Feiler}
\date{Algorytmy i Struktury Danych gr. LJ*, 14 maja 2018 roku}

\newcommand\textdoubleplus{+\kern-1.3ex+\kern0.8ex}
\newcommand\doubleplus{\ensuremath{\mathbin{+\mkern-10mu+}}}

\begin{document}

\begin{frame}
  \titlepage
\end{frame}

\begin{frame}{Outline}
  \tableofcontents
\end{frame}

\section{Wstęp}

\subsection{Problem tablic asocjatywnych uporządkowanych}

\begin{frame}{Wstęp --- problem tablic asocjatywnych uporządkowanych}
	Problem implementacji tablic asocjatywnych, a więc, optymalnej struktury danych dla
    zbioru wartości typu $V$, mając funkcję klucza
    $k : V \rightarrow K$ (różnowartościową lub nie (wielozbiór kluczy)), 
    gdzie $K$ jest pewnym typem klucza, pod kątem
    efektywnego znajdowania wartości po kluczu, 
    tzn. po odpowiadającej im wartości funkcji $k$.
    
    \vspace{0.25cm}
    
    Problem implementacji tablic asocjatywnych uporządkowanych, a więc rozszerzenie
    problemu tablicowania o zagadnienie optymalizacji pod kątem odwiedzania
    elementów w porządku, mając dodatkowo relację porządku liniowego $\leq$ na $K$ —
    która gdy złożona z $k$ tworzy relację na $V$:
    \begin{itemize}
    \item spójnego praporządku $\lesssim$ ~ 
    (przechodnią, spójną, i zwrotną) ~ 
    ($k$ nie musi być injekcją, jest multizbiór kluczy),
    \item która, jeśli $k$ injekcją, jest również porządkiem liniowym 
    (dodatkowo słabo antysymetryczną) ~ (tylko zbiór kluczy).
    \end{itemize}
\end{frame}
\subsection{Problem implementacji typu zbiorowego i wielozbiorowego}
\begin{frame}{Wstęp --- problem impl. typu zbiorowego i wielozbiorowego}
Problem implementacji typu zbiorowego lub multizbiorowego uporządkowanego,
a więc, optymalnej struktury danych dla zbioru wartości typu $V$,
mając relację spójnego praporządku $\lesssim$ na $V$ ~\\ (przechodnią, spójną i zwrotną) ~ \\
(w przypadku typu zbiorowego, 
$\lesssim$ jest również porządkiem liniowym\\ --- dodatkowo słabo antysymetryczną ---),\\
pod kątem odwiedzania elementów w porządku ~\\
(a także, w przypadku typu zbiorowego, efektywnego pomijania duplikatów przy dodawaniu
wartości do struktury).

Jeżeli mamy też pewną funkcję $k : V \rightarrow K$, gdzie $K$ jest pewnym typem
na którym mamy określony pewien porządek liniowy $\leq$, \\
który złożony z $k$ jest równoważny $\lesssim$,
problem można sprowadzić do problemu implementacji tablic asocjatywnych uporządkowanych.
\end{frame}

\subsection{Drzewa przeszukiwania}

\begin{frame}{Wstęp --- drzewa przeszukiwania}
Drzewa przeszukiwania są jednym z rozwiązań 
problemu implementacji tablic asocjatywnych uporządkowanych
lub typów (wielo)zbiorowych.

\vspace{0.3cm}
Są to struktury--drzewa \\
z parametryzacją rozmieszczenia poziomego (uporządkowane),
\begin{itemize}
\item która jest unikalna na zbiorze poddrzew danego węzła, 
\item dla której wartości ustalona spójna relacja porządku
implikuje relację $\lesssim$ zarówno dla wierzchołków poddrzew
po odpowiednich stronach jak i dla pozostałych wartości w danym węźle
(czyli relację $\leq$ dla kluczy).
\end{itemize}
\end{frame}

\subsection{Binarne drzewa przeszukiwania}

\begin{frame}{Wstęp --- binarne drzewa przeszukiwania}
Najprostszą odmianą drzew przeszukiwania są binarne drzewa przeszukiwania (BST).
Są to drzewa przeszukiwania, w których
\begin{itemize}
\item węzły drzewa zawierają tylko jedną wartość,
\item mają każdy od zera do dwóch poddrzew, 
\item ich parametryzacja rozmieszczenia poziomego jest 
\begin{itemize}
\normalsize
\item dwuwartościowa (boolowska)
\item o wartości spełniającej implikację zachodzenia relacji $\lesssim$ między wartościami
danego węzła a wierzchołka danego jego poddrzewa
\begin{itemize}
\normalsize
\item jeżeli $\lesssim$ jest porządkiem liniowym to równej zachodzeniu relacji $\lesssim$,\\
\item wówczas również, przez przechodniość tej relacji, zachodzeniu relacji $\lesssim$
również z pozostałym poddrzewem,
jeżeli są dwa pod danym węzłem.
\end{itemize}
\end{itemize}
\end{itemize}

\end{frame}

\subsubsection{Działania na BST}

\begin{frame}{Działania na BST --- struktury algebraiczne}
\begin{flushleft}
\begin{flalign*}
\shortintertext{Przedstawmy drzewo BST jako typ algebraiczny}
\texttt{class}&~\textit{POrd}~K~\texttt{where} ~ ~
 (\leq) : K \rightarrow K \rightarrow \textit{Bool}&\\
\texttt{class}&~\textit{TPreOrd}~V~\texttt{where} ~ ~
 (\lesssim) : V \rightarrow V \rightarrow \textit{Bool}\\
\texttt{class}&~(\textit{TPreOrd}~V, \textit{POrd}~K)\Rightarrow
\textit{NodeVal}~V~K~\texttt{where} ~
 k : V \rightarrow K\\
\end{flalign*}
\begin{flalign*}
&\texttt{type}~\textit{TwoSub}~V~K=(\textit{Maybe}~(\textit{Node}~V~K)~,~\textit{Maybe}~(\textit{Node}~V~K))&\\
&\texttt{data}~\textit{Node}~V~K~\texttt{where}&\\
& ~ ~ ~ \textit{Node} : (\textit{NodeVal}~V~K) \Rightarrow 
V \rightarrow \textit{TwoSub}~V~K \rightarrow \textit{Node}~V~K \\
\end{flalign*}
\end{flushleft}
\end{frame}

\begin{frame}{Działania na BST --- znajdowanie po kluczu oraz traversal}
\begin{flalign*}
&\textit{get}:(\textit{Eq}~K)\Rightarrow\textit{Maybe}~(\textit{Node}~V~K)
\rightarrow K\rightarrow [V]&\\
&\textit{get}~\textit{Nothing}~\_=[]&\\
&\textit{get}~(\textit{Just}~(\textit{Node}~t~(l,p)))~x=&\\
& ~ ~ ~ [t ~|~ k~t == x] \doubleplus 
(\textit{get}~ ~(\texttt{if}~x\leq k~t~ ~\texttt{then}~l~ ~\texttt{else}~p)~ ~x)&\\
&\textit{list}:\textit{Maybe}~(\textit{Node}~V~K)\rightarrow[V]&\\
&\textit{list}~\textit{Nothing}=[]&\\
&\textit{list}~(\textit{Just}~(\textit{Node}~t~(l,p))= \textit{list}~l \doubleplus [t]
\doubleplus \textit{list}~p &\\
\end{flalign*}
\end{frame}

\begin{frame}{Działania na BST --- wstawianie}
\begin{flalign*}
\textit{insert}&:(\textit{NodeVal}~V~K)\Rightarrow\textit{Maybe}~(\textit{Node}~V~K)
\rightarrow v\rightarrow\textit{Node}~V~K\\
\textit{insert}&~\textit{Nothing}~x=\textit{Node}~x~(\textit{Nothing},\textit{Nothing})\\
\textit{insert}&~(\textit{Just}~ ~(\textit{Node}~t~(l,p)))~x\\
\texttt{{-}{-}}~&|~ ~ ~x\lesssim t ~ \land ~ t\lesssim x ~=\textit{Node}~t~(l,p)~ ~ ~
\texttt{{-}{-}}~\text{el. duplikatów}\\
 &|~ ~ ~x\lesssim t ~=\textit{Node}~t~(\textit{Just}~(\textit{insert}~l~x)~,~p)\\
 &|~ ~ ~\texttt{otherwise} ~=\textit{Node}~t~(l,~\textit{Just}~(\textit{insert}~p~x))
\end{flalign*}
\end{frame}

\begin{frame}{Działania na BST --- \texttt{popLeftmost}}
\begin{flalign*}
&\texttt{type}~\textit{PopFun}~V~K=\textit{Node}~V~K\rightarrow 
(\textit{Maybe}~(\textit{Node}~V~K)~,~V)&\\
&\textit{popLeftmost}:\textit{PopFun}~V~K&\\
&\textit{popLeftmost}~(\textit{Node}~t~(\textit{Nothing},p))=(p,t)&\\
&\textit{popLeftmost}~(\textit{Node}~t~(\textit{Just}~l,~p))=&\\
&~ ~ ~\texttt{let}~(n,v)=\textit{popLeftmost}~l~\texttt{in}~
(\textit{Just}~(\textit{Node}~t~(l,n))~,~v)&\\
&\textit{popRightmost}:\textit{PopFun}~V~K&\\
&\textit{popRightmost}~(\textit{Node}~t~(l,\textit{Nothing}))=(l,t)&\\
&\textit{popRightmost}~(\textit{Node}~t~(l,\textit{Just}~p))=&\\
&~ ~ ~\texttt{let}~(n,v)=\textit{popRightmost}~p~\texttt{in}~
(\textit{Just}~(\textit{Node}~t~(n,p))~,~v)&\\
\end{flalign*}
\end{frame}

\begin{frame}{Działania na BST --- \texttt{popDeeperOneOffTop}}
\begin{flalign*}
&\textit{popDeeperOneOffTop}:\textit{PopFun}~V~K&\\
&\textit{popDeeperOneOffTop}~(\textit{Node}~t~\textit{Nothing}~\textit{Nothing})=
(\textit{Nothing}~,~t)&\\
&\textit{popDeeperOneOffTop}~(\textit{Node}~t~l~p)=&\\
& ~ \texttt{let}~((n_l,n_p),v)=\textit{\_popDeeperOneOff}~(l~,~p)~\texttt{in}~
(\textit{Just}~(\textit{Node}~t~n_l~n_p))&\\
&\textit{\_popDeeperOneOff}:\textit{TwoSub}~V~K\rightarrow (\textit{TwoSub}~V~K~,~V)&\\
&\textit{\_popDeeperOneOff}~(
\textit{Just}~(\textit{Node}~t_l~(o_l,l)),
\textit{Just}~(\textit{Node}~t_p~(p,o_p))=\\
&~ ~ ~ ~ ~ \texttt{let}~((n_l,n_p),v)=\textit{\_popDeeperOneOff}~(l,p)~
\texttt{in}&\\
&~ ~ ~ ~ ~ ~ ~ (\textit{Just}~(\textit{Node}~t_l~(o_l,n_l)),
\textit{Just}~(\textit{Node}~t_p~(n_p,o_p))),v)&\\
&\textit{\_popDeeperOneOff}~(\textit{Just}~l~,\textit{Nothing})=\\
&~ ~ ~ ~ ~ \texttt{let}~(n,v)=\textit{popRightmost}~l~\texttt{in}~((n,\textit{Nothing}),v)&\\
&\textit{\_popDeeperOneOff}~(\textit{Nothing},\textit{Just}~p)=\\
&~ ~ ~ ~ ~ \texttt{let}~(n,v)=\textit{popLeftmost}~p~\texttt{in}~((\textit{Nothing},n),v)&\\
&\textit{\_popDeeperOneOff}~...
\end{flalign*}
\end{frame}

\begin{frame}{Działania na BST --- \textit{\_}\texttt{popDeeperOneOff}}
\begin{flalign*}
& ~ \texttt{let}~((n_l,n_p),v)=\textit{\_popDeeperOneOff}~(l~,~p)~\texttt{in}~
(\textit{Just}~(\textit{Node}~t~n_l~n_p))&\\
&\textit{\_popDeeperOneOff}:\textit{TwoSub}~V~K\rightarrow (\textit{TwoSub}~V~K~,~V)&\\
&\textit{\_popDeeperOneOff}~(
\textit{Just}~(\textit{Node}~t_l~(o_l,l)),
\textit{Just}~(\textit{Node}~t_p~(p,o_p))=\\
&~ ~ ~ ~ ~ \texttt{let}~((n_l,n_p),v)=\textit{\_popDeeperOneOff}~(l,p)~
\texttt{in}&\\
&~ ~ ~ ~ ~ ~ ~ (\textit{Just}~\$~\textit{Node}~t_l~(o_l,n_l),
\textit{Just}~\$~\textit{Node}~t_p~(n_p,o_p)),v)&\\
&\textit{\_popDeeperOneOff}~(\textit{Just}~l~,\textit{Nothing})=\\
&~ ~ ~ ~ ~ \texttt{let}~(n,v)=\textit{popRightmost}~l~\texttt{in}~((n,\textit{Nothing}),v)&\\
&\textit{\_popDeeperOneOff}~(\textit{Nothing},\textit{Just}~p)=\\
&~ ~ ~ ~ ~ \texttt{let}~(n,v)=\textit{popLeftmost}~p~\texttt{in}~((\textit{Nothing},n),v)&\\
&\textit{\_popDeeperOneOff}~(\textit{Nothing},\textit{Nothing})=\texttt{undefined}&\\
\end{flalign*}
\end{frame}

\begin{frame}{Działania na BST --- usuwanie elementów po kluczu}
\begin{flalign*}
\textit{delete}&:(\textit{Eq}~K)\Rightarrow\textit{Maybe}~(\textit{Node}~V~K)
\rightarrow K\rightarrow\textit{Maybe}~(\textit{Node}~V~K)&\\
\textit{delete}&~\textit{Nothing}~\_ = \textit{Nothing}&\\
\textit{delete}&~(\textit{Just}~(\textit{Node}~t~(l,p)~)~)~ ~ x&\\
&|~ ~ ~k~t==x ~\land~ \textit{isNothing}~l ~= \textit{delete}~p~x &\\
&|~ ~ ~k~t==x ~\land~ \textit{isNothing}~p ~= \textit{delete}~l~x &\\
&|~ ~ ~k~t==x ~=\textit{flip delete}~x~\$~\textit{Just}~\$ &\\
&~ ~ ~ ~ ~ ~\texttt{let}~((n_l,n_p),v) = \textit{\_popDeeperOneOff}~(l,p) &\\
&~ ~ ~ ~ ~ ~ ~ ~\texttt{in}~\textit{Node}~v~(n_l,n_p)&\\
\end{flalign*}
\end{frame}

\subsection{Niezrównoważone BST są nieefektywne}

\begin{frame}{Niezrównoważone BST są nieefektywne}
\begin{figure}
\begin{tikzpicture}[level distance=10mm]
\node [circle, draw] (a){$1$}
	child {node [midway] (bb) {} edge from parent[draw=none]}
	child {node [circle, draw] (b){$2$}
    	child {node [midway] (cc) {} edge from parent[draw=none]}
        child {node [circle, draw] (c) {$3$}
        	child {node [midway] (dd) {} edge from parent[draw=none]}
            child {node [circle, draw] (d) {$4$}
            	child {node [midway] (ee) {} edge from parent[draw=none]}
                child {node [circle, draw] (e) {$5$}}}}};
\end{tikzpicture}
\caption{\label{fig:sequenceunbalancedbst}
Niezbalansowane BST sekwencji kilku kolejnych liczb nat.}
\end{figure}
\end{frame}

\section{B-drzewa pokrótce}

\subsubsection{Ogólnik}

\begin{frame}{B-drzewa --- ogólnik}
	B-drzewa (znaczenie \textit{B} jest nieustalone) są poniekąd uogólnieniem BST.
    Są do drzewa przeszukiwania, w których
    \begin{itemize}
    	\item węzły zawierają od $d$ do $2d$ wartości, gdzie $d\in\mathbb{N}^+$ ustalone
        \item jeżeli dany węzeł zawiera $n$ wartości, to ma $n+1$ poddrzew
        \item ich parametryzacja rozmieszczenia poziomego jest
        \begin{itemize}
        	\item $n$-wartościowa, z ustalonym dla tych wartości ostrym porządkiem
            liniowym $<$, przy czym dla każdych dwóch elementów zbioru wartości parametryzacji 
            $n+1$-wartościowej wartość znajdująca się w zb. w. parametryzacji
            $n$-wartościowej będzie zawsze w relacji $<$ z wartością która się w
            nim nie znajduje
        \end{itemize}
    \end{itemize}
\end{frame}

\begin{frame}{B-drzewa --- definicja wg Knutha}
	.
\end{frame}

\begin{frame}{B-drzewa --- implementacja}
\begin{minipage}[t]{0.48\linewidth}
	\inputminted{c}{bdrzewa-impl1.c}
\end{minipage}
\begin{minipage}[t]{0.48\linewidth}
	\inputminted{c}{bdrzewa-impl2.c}
\end{minipage}
\end{frame}

%\vskip 1cm
%
%\begin{block}{Examples}
%Some examples of commonly used commands and features are included, to help you get started.
%\end{block}

%\section{Some \LaTeX{} Examples}

%\subsection{Tables and Figures}

%\begin{frame}{Tables and Figures}

%\begin{itemize}
%\item Use \texttt{tabular} for basic tables --- see Table~\ref{tab:widgets}, for example.
%\item You can upload a figure (JPEG, PNG or PDF) using the files menu. 
%\item To include it in your document, use the \texttt{includegraphics} command (see the comment below in the source code).
%\end{itemize}

%% Commands to include a figure:
%%\begin{figure}
%%\includegraphics[width=\textwidth]{your-figure's-file-name}
%%\caption{\label{fig:your-figure}Caption goes here.}
%%\end{figure}

%\begin{table}
%\centering
%\begin{tabular}{l|r}
%Item & Quantity \\\hline
%Widgets & 42 \\
%Gadgets & 13
%\end{tabular}
%\caption{\label{tab:widgets}An example table.}
%\end{table}

%\end{frame}

\end{document}
